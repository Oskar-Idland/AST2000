\documentclass[reprint,english,notitlepage]{revtex4-2}
\usepackage{amsmath}
\usepackage[mathletters]{ucs}
\usepackage[utf8x]{inputenc}
\usepackage[english]{babel}
\usepackage{esint}
\usepackage{physics,amssymb}
\usepackage{graphicx}
\usepackage{xcolor}
\usepackage{hyperref}
\usepackage{listings}
\usepackage{subfigure}
% \usepackage[style=science, backend=biber]{biblatex}
% \addbibresource{Report_Part_5.bib} TODO: Slett før innlevering
\hypersetup{
    colorlinks,
    linkcolor={red!50!black},
    citecolor={blue!50!black},
    urlcolor={blue!80!black}}

\lstset{inputpath=,
    backgroundcolor=\color{white!88!black},
    basicstyle={\ttfamily\scriptsize},
    commentstyle=\color{magenta},
    language=Python,
    morekeywords={True,False},
    tabsize=4,
    stringstyle=\color{green!55!black},
    frame=single,
    keywordstyle=\color{blue},
    showstringspaces=false,
    columns=fullflexible,
    keepspaces=true}

\begin{document}

\title{Satellite Launch}
\author{Oskar Idland \& Jannik Eschler}
\date{\today}
\affiliation{Institute of Theoretical Astrophysics, University of Oslo}

\begin{abstract}
    This is an abstract \colorbox{red}{Complete this summary at the end of the paper}
\end{abstract}
\maketitle

\section{Introduction} \label{sec:introduction}
The purpose of this project is the launch of our shuttle.
Interplanetary travel have huge cost and risks.
Therefore, we must guarantee success by planning ahead of our journey.
We will develop a simulation to visualize our orbit given some parameters as a means to get a good picture of where we will end up.\\

After completing these final preparations and simulations, we will then send our spacecraft towards its destination.
The launch and interplanetary travel will be executed based on the calculations and simulations we have done in the previous parts of this series of reports.
However, as we have done some assumptions and simplifications in our simulations, we expect some deviations in the actual path of the spacecraft.
This is due to factors such as solar winds, gravitational forces from small objects and friction.\\
To reach our destination, we will therefore launch the rocket on the simulated trajectory and do some corrections of our trajectory, if the actual trajectory deviates from the simulated trajectory.\\
These corrections will be done by firing our rocket engine to change our velocity, and therefore our trajectory as described in figure~\ref{fig:boost_fig}.

\begin{figure}[h]
    %% H(Here), h(here approx), t(top of page), b(bottom of page)
    \centering
    \includegraphics[scale=0.12]{Figures/boost_fig}
    \caption{Visualisation of a correctional boost to change the trajectory.}\label{fig:boost_fig}
\end{figure}

$  $
The goal is to get close enough to the planet so that the gravitational forces from the planet dominate over the gravitational forces from the rest of the planets and the star.
We can then initialise an orbit injection manoeuvre to enter the orbit around the planet.\\

When in orbit around the destination planet, we will have to orient ourselves and analyze the orbit we are in.
Using the onboard instruments we will determine all necessary parameters to be able to fully simulate the orbit.
This is to predict the movement and position of our spacecraft at a later point in time, which will be necessary to determine a time and position for the landing of our rover.



\section{Theory} \label{sec: theory}
A description of the Leapfrog simulation method can be found in %~\parencite[][]{part2} TODO: Delete Comment.

\section{Method} \label{sec: method}
\subsection{Simulating trajectory} \label{ssec: simulating trajectory}
To simulate the trajectory our shuttle we will calculate all the forces acting upon it. Using Newton's second law of motion we get this expression for the acceleration the shuttle will experience on its journey. 
\[
\mathbf{a} = - G \frac{M_{\text{star}}}{\mathbf{\left\vert \mathbf{r} \right\vert ^{3}}} \mathbf{r} - \sum_{i} G \frac{m_i}{\left\vert \mathbf{r} - \mathbf{r}_i \right\vert ^{3}} \left( \mathbf{r} - \mathbf{r}_i \right) 
\]
Starting from the left, $ \mathbf{a} $ is the acceleration of the shuttle, $ G $ the gravitational constant, $ M_{\text{star}} $ the mass of the star and $ \mathbf{r} $ is the position of the shuttle. We then add the sum of the acceleration from the planets in the solar system where $ m_i $ and $ \mathbf{r}_i $ is the mass and position of each planet respectively. The shuttle has an insignificant amount of mass and we will ignore it during our calculations. 

Then we use the Leapfrog integration method get the next position after a small amount of time $  Δt  $ has passed as shown in the equations below

\begin{subequations} \label{eq: leapfrog integration}
    \begin{equation}
        v_{i + \frac{1}{2}} = v_i + a_i \frac{Δt }{2}
    \end{equation}
    \begin{equation}
      x_{i+1} = x_{i} + v_{i + \frac{1}{2}} Δt 
    \end{equation}
    \begin{equation}
      v_{i+1} = v_{i + \frac{1}{2}} + a_{i+1} \frac{Δt }{2}
    \end{equation}
\end{subequations}

% Metode som kan brukes om ting blir simulert som N-body system. Hvis ikke bruk det som står over
% shuttle will have we simplify our system into a N-body system where all the planets and star will be combined into a single body. As the mass of the shuttle is so small in comparison to the rest of our solar system, we will disregard its gravitational pull. First we will calculate the center of mass $ \mathbf{CM} $
% \[
% \mathbf{CM} = \frac{1}{M} \sum_{i} m_i \mathbf{r_i}
% \]
% where $ M $ is the total mass of our solar system and $ m_i $ and $ r_i $ is the mass and position of each planet respectively. We then find the total momentum $ \mathbf{p} $ as a sum of all the momentum of each planet in the system
% \[
% \mathbf{p} = \sum_{i} m_i \mathbf{v_i}
% \]
% where $ \mathbf{v_i} $ is the velocity of each planet. 
% As we use the position of the star as origin we represent the position of the $ \mathbf{CM} $ as $ \mathbf{r}_{sys} $. This position changes over time with a velocity $ \mathbf{v}_{sys} = \frac{\mathbf{p}}{M} $.  
% As the shuttle travels, its position $ \mathbf{r} $ will be influenced by its initial velocity $ \mathbf{v_0} $ and the gravitational pull of the system. The acceleration of the spacecraft is given by 
% \[
% \mathbf{a} = -G \frac{M}{\left\vert \mathbf{r} - \mathbf{r}_{sys} \right\vert ^{3}} (\mathbf{r} - \mathbf{r}_{sys}).
% \]
We will use this method during the launch of our rocket.

\subsection{Getting close enough to the target planet}
To make our journey as easy as possible we first try to check where our home planet and target planet are the closest. Using the orbits, calculated from the second paper %~\parencite[][]{part2} TODO: Delete comment 
of this series of papers, we iterate over all positions and check at which time $ t_{0} $ the distance is the smallest. This is the time we will begin our launch. The next step is to calculate the optimal direction and speed of our initial velocity $ \mathbf{v}_0 $. Our initial position $ \mathbf{r}_0 $ will have the same directional vector meaning $ \hat{\mathbf{v}}_0 = \hat{\mathbf{r}}_0  $. To figure out the direction to launch our planet we begin by using an educated guess by pointing the shuttle directly at the target planet. The simulation will continuously check if we get a small enough distance $ d $ to allow us to begin a stable orbit. This distance must be smaller than $ l $ given by
\[
d \le l, \qquad l = \left\vert \mathbf{r} \right\vert \sqrt{\frac{M_{target}}{M_{star}}}.
\]


To make the parameters of our simulation easier we switch to polar coordinates $ (r, θ) $. We will iterate over multiple cases of possible angles and speed. This will be done by choosing a median angle and speed. Then have a variance which we will subtract for the lower end of angles and velocities, and add for our upper end. Evenly spaced out values in this interval will be used to get a wide range of possible values. In our case as seen in figure \ref{fig: closest orbit}, we get the closest orbit in the third quadrant in terms of polar coordinates. We begin our simulation by having a median angle of $ \frac{5}{4}π $ with a variance of $ \frac{π}{4} $. This let us cover the entire third quadrant. As seen in figure \ref{fig: simulation scattered} the trajectory will be quite scattered. Next we will show some examples of how we will narrow down both the trajectory and velocity to get close enough to our target planet. 

\begin{figure}[h!]
  \centering
  \includegraphics[scale = .125]{Figures/full_angle_scatter.pdf}
  \caption{Simulation of trajectory with relatively wide range of angles. The dots represent were the shuttle was the closest to the target planet }
  \label{fig: simulation scattered}
\end{figure}
In our case it seems our median angle was quite close. Therefore we only tighten up the span of angles by reducing the variance of angles. We then get the following result in figure \ref{fig: simulation tighter}. 

\begin{figure}[h!]
  \centering
  \includegraphics[scale = .125]{Figures/narrowing_down_angle.pdf}
  \caption{Simulation of trajectory after narrowing down the variance}
  \label{fig: simulation tighter}
\end{figure}

This is how we will find better and better values for our angle. When it comes to the speed we will use the same technique. As seen in the example from figure \ref{fig: pre speed increase} our speed was too low in all three cases. 

\begin{figure}[h!]
  \centering
  \includegraphics[scale = .125]{Figures/pre_speed_increase.pdf}
  \caption{Simulation of trajectory before speed increase}
  \label{fig: pre speed increase}
\end{figure}

Then it makes sense to increase it and check if we get closer. When the speed was increased we got even closer to our target as seen in figure \ref{fig: post speed increase}

\begin{figure}[h!]
  \centering
  \includegraphics[scale = .125]{Figures/post_speed_increase.pdf}
  \caption{Simulation of trajectory after speed increase}
  \label{fig: post speed increase}
\end{figure}

These examples were to illustrate how we can refine our angles and velocity to get close enough to try and orbit the target planet. For illustrative purposes we only used 5 different angles and three different speed, which gave us 15 different trajectories. In reality we will use hundreds of both to find the answer with as few adjustments as possible.
The simulation only yields the velocity the shuttle need in reference to our star in origin. We subtract the velocity of our home planet from the calculated value to get the actual value our shuttle will need. 


\subsection{Launching the Spacecraft}\label{subsec:launching-the-spacecraft}
    Having simulated the necessary parts of the launch, journey and conditions on our destination planet, we can now send the spacecraft on its journey.

\newpage
\section{Results} \label{sec: results}
\begin{figure}[h!]
  \centering
  \includegraphics[scale = .125]{Figures/closest_orbit.pdf}
  \caption{Position where the target planet and our home planet were the closest}
  \label{fig: closest orbit}
\end{figure}

The simulation of the trajectory yielded a necessary speed of 12 AU/yr and an angle of $ \colorbox{red}{X} ^{o} $ degrees with a travel time of $ \colorbox{red}{Y} $ years. We launched our rocket approximately in the year \colorbox{red}{Z}. After adjusting for our home planets own velocity we get an initial velocity $ \mathbf{v}_0 = \colorbox{red}{$(v_1, v_2) $} $ and a speed of $ \left\vert \mathbf{v}_0 \right\vert = \colorbox{red}{$ v_0 $} \  AU / yr$. Our initial position $ \mathbf{r}_0 $ will then be $ \mathbf{r}_0 = R \hat{\mathbf{v}} + \mathbf{r}_{\text{hp}} $, where $ R $ is the radius and $ \mathbf{r}_{\text{hp}} $ the position of our home planet. 
\begin{figure}[h!]
  \centering
  \includegraphics[scale = .125]{Figures/good_enough_distance.pdf}
  \caption{Position of shuttle and target planet at optimal distance}
  \label{fig: good enough distance}
  \colorbox{red}{Update when the actual trajectory has been chosen}
\end{figure} 


\section{Discussion} \label{sec: discussion}


\section{Conclusion} \label{sec: conclusion}

\section{Appendix} \label{sec: appendix}

\section*{ACKNOWLEDGMENTS}

\newpage
% \printbibliography TODO: Remove Comment
\end{document}