\documentclass[reprint,english,notitlepage]{revtex4-2}
\usepackage{amsmath}
\usepackage[mathletters]{ucs}
\usepackage[utf8x]{inputenc}
\usepackage[english]{babel}
\usepackage{esint}
\usepackage{physics,amssymb}
\usepackage{graphicx}
\usepackage{xcolor}
\usepackage{hyperref}
\usepackage{listings}
\usepackage{subfigure}
% \usepackage[style=science, backend=biber]{biblatex}
% \addbibresource{References_Part_4.bib} TODO: Slett før innlevering
\hypersetup{
    colorlinks,
    linkcolor={red!50!black},
    citecolor={blue!50!black},
    urlcolor={blue!80!black}}

\lstset{inputpath=,
    backgroundcolor=\color{white!88!black},
    basicstyle={\ttfamily\scriptsize},
    commentstyle=\color{magenta},
    language=Python,
    morekeywords={True,False},
    tabsize=4,
    stringstyle=\color{green!55!black},
    frame=single,
    keywordstyle=\color{blue},
    showstringspaces=false,
    columns=fullflexible,
    keepspaces=true}

\begin{document}

\title{Satellite Launch}
\author{Oskar Idland \& Jannik Eschler}
\date{\today}
\affiliation{Institute of Theoretical Astrophysics, University of Oslo}

\begin{abstract}
    This is an abstract \colorbox{red}{Complete this summary at the end of the paper}
\end{abstract}
\maketitle

\section{Introduction} \label{sec:introduction}
The purpose of this project is the launch of our shuttle. Interplanetary travel have huge cost and risks. Therefore we must guarantee success by planning ahead of our journey. We will develop a simulation to visualize our orbit given some parameters as a means to get a good picture of where we will end up. 

\section{Theory} \label{sec: theory}
\colorbox{red}{Refferer til forklaring av leapfrog fra tidligere deler}

\section{Method} \label{sec: method}



\section{Results} \label{sec: results}

\section{Discussion} \label{sec: discussion}

\section{Conclusion} \label{sec: conclusion}

\section{Appendix} \label{sec: appendix}

\section*{ACKNOWLEDGMENTS}

\section*{References} \label{sec: references}

\end{document}