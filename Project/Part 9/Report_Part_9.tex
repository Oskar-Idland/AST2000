\documentclass[reprint,english,notitlepage]{revtex4-2}
\usepackage{amsmath}
\usepackage[mathletters]{ucs}
\usepackage[utf8x]{inputenc}
\usepackage[english]{babel}
\usepackage{esint}
\usepackage{physics,amssymb}
\usepackage{graphicx}
\usepackage{xcolor}
\usepackage{hyperref}
\usepackage{listings}
\usepackage{subfigure}
\usepackage[style=science, backend=biber]{biblatex}
\addbibresource{Report_Part_9.bib}
\hypersetup{
    colorlinks,
    linkcolor={red!50!black},
    citecolor={blue!50!black},
    urlcolor={blue!80!black}}

\lstset{inputpath=,
    backgroundcolor=\color{white!88!black},
    basicstyle={\ttfamily\scriptsize},
    commentstyle=\color{magenta},
    language=Python,
    morekeywords={True,False},
    tabsize=4,
    stringstyle=\color{green!55!black},
    frame=single,
    keywordstyle=\color{blue},
    showstringspaces=false,
    columns=fullflexible,
    keepspaces=true}

\begin{document}

\title{Interplanetary Rocket Launch}
\author{Candidates: 15369 \& 15401}
\date{\today}
\affiliation{Institute of Theoretical Astrophysics, University of Oslo}

\begin{abstract}
    \colorbox{red}{Write Abstract}
\end{abstract}
\maketitle

\section{Exercise 1}\label{sec:exercise-1}
    \subsection{Introduction}\label{subsec:introduction1}
        In this exercise, we will be looking at the gravitational doppler shift.
        This is an effect, where the gravitational deformation of spacetime affects the wavelength of light, so that it appears either longer or shorter.\\
        For an observer close to a large mass, time will go at a different rate than for a far-away observer.
        This time difference, affects the frequency and therefore wavelength of light.
        We will therefore try to describe this difference in wavelength for an observer close to the mass and the observer far away.

    \subsection{Situation}\label{subsec:situation1}
        We will be looking at our own solar system containing the sun and the earth.
        To simplify things, we will be using polar coordinates.\\
        One observer will be positioned on the surface of the sun, whereas another observer will be positioned far away from the sun.
        This allows us to measure the wavelength of the light perceived by both observers.\\
        We will then send out light, or electromagnetic waves, radially outward from the mass or inwards to the mass.
        This way, we are able to simplify our calculations and neglect tangential components.

    \subsection{Method}\label{subsec:method1}
        According to general relativity, the time interval $\Delta t_{shell}$ between the two peaks of the electromagnetic wave varies when spacetime is deformed, such as in the presence of a large mass.\\
        To be able to relate the time intervals $\Delta t$, which is the time interval seen by an observer far away, and $\Delta t_{shell}$, which is the time interval for the shell observer, we will be using the formula for the Schwarzschild line element ~\eqref{schwarzschild_line} between two ticks $A$ and $B$ on the clock of the sell observer.
        \begin{align}
            \Delta s^2_{AB} = \left(1-\frac{2M}{r}\right) \Delta t^2 - \frac{\Delta r^2_{AB}}{\left(1-\frac{2M}{r}\right)} - r^2 \Delta \phi^2_{AB} \label{schwarzschild_line}
        \end{align}
        By using the fact that the shell observer is at rest, equation~\eqref{schwarzschild_line} can be simplified, and $\Delta s_{AB}$ can be substituted by $\Delta t_{shell}$.
        This yields the formula
        \begin{align}
                \Delta t = \frac{\Delta t_{shell}}{\sqrt{\left(1-\frac{2M}{r}\right)}} \label{eq2}
        \end{align}
        The derivation of this formula can be found in section ~\ref{subsec:derivation-of-time-interval-difference}\\\\

        This formula for the difference in time intervals for the observers, can then be used to find the gravitational Doppler shift of light, which is close to a large mass.
        We know that the doppler shift is given by formula~\eqref{grav_dopplershift1}.
        \begin{align}
            \frac{\Delta \lambda}{\lambda_{shell}} \label{grav_dopplershift1}
        \end{align}
        Since we define the relationship of the frequency $\nu$ and the wavelength $\lambda$ to be $\nu = \frac{c}{\lambda}$ and $\nu = \frac{1}{\Delta t}$, we can substitute $\lambda$ by $\Delta_t$ and $\lambda_{shell}$ by $\Delta_t_{shell}$ in equation ~\eqref{grav_dopplershift1}.
        By then inserting the expression found in ~\eqref{eq2}, we get
        \begin{align}
            \frac{\Delta \lambda}{\lambda_{shell}} &= \frac{\Delta t - \Delta t_{shell}}{\Delta t_{shell}}\\
            \frac{\Delta \lambda}{\lambda_{shell}} &= \frac{1}{\sqrt{\left(1-\frac{2M}{r}\right)}} - 1
        \end{align}\\\\

        We can see that if the distance $r \gg 2M$, the second term under the root, will be going towards zero.
        We define $x = \frac{2M}{r}$ and create a taylor expansion around $x = 0$.
        Using a taylor expansion of degree 1, we find that the doppler shift can be approximated by
        \begin{align}
            \frac{\Delta \lambda}{\lambda_{shell}} = \frac{M}{r}
        \end{align}
        For the derivation, see section~\ref{subsec:exercise-1.3}.
        


    \subsection{Conclusion}\label{subsec:conclusion1}
        Looking at the results, we can see that there are significant effects from general relativity when we are close to a large mass.\\
        By using electromagnetic radiation, we determined the difference in proper time for the observer close to the mass and the far-away observer, and its effects.\\

        When sending electromagnetic radiation radially outwards from the mass, the wavelength of the radiation, will appear increased for a far-away observer or redshifted.
        The opposite effect happens when radiation is sent inwards to the mass, which is called blueshift.
        The magnitude of this effect depends on the mass and the radius from where the radiation is sent out or received.




\section{Exercise 2}\label{sec:exercise-2}
    \subsection{Introduction}\label{subsec:introduction2}
        To get a better grasp of the movement of an object when taking general relativity into account, we will be taking a closer look at the Schwarzschild line element.
        Together with the principle of maximum aging, we will be able to derive a quantity, which is equivalent to mechanical spin.\\
        This helps us to further understand movement, and calculations of the movement in free float.

    \subsection{Situation}\label{subsec:situation2}
        The main object of interest in this exercise is an object, which is orbiting around a black hole with mass $M$ and radius $2M$.
        The orbit does not need to be a circular orbit, but can have an elliptical shape.\\
        We define three points along the orbit of the object (Note that we will be using a polar coordinate system in this exercise).
        Point $1$ at time $t_1$ and position $(r_1, \phi_1)$, point $2$ at time $t_2$ and position $(r_2, \phi_2)$ and point $3$ at time $t_3$ and position $(r_3, \phi_3)$.
        We assume all parameters to be known, except $\phi_2$, and can then define two time steps $\Delta t_{12} = (t_2-t_1)$ and $\Delta t_{23} = (t_3-t_2)$.\\

        In the following derivations and calculations, the time steps $\Delta_t_{12}$ and $\Delta t_{23}$ are extremely small, which will allow us to use Lorentz-transformation and therefore the Schwarzschild line element.
        Therefore, we will assume the radius during each interval to be constant and equal to respectively $r_A$ and $r_B$.\\

        The proper time interval corresponding to $\Delta t_{12}$ and $\Delta t_{23}$ are $\Delta\tau_{12}$ and $\Delta\tau_{23}$.

    \subsection{Method}\label{subsec:method2}
        The change in position corresponding to free float, is found when minimising the change in spatial coordinates, and therefore maximizing the change in time coordinates.
        This is called the principle of maximum aging.
        Finding these changes in coordinates, is a relatively straight forward, but requires some calculations.
        We will first need to approximate $\Delta s_{12}$ and $\Delta s_{23}$ by $\Delta \tau_{12}$ and $\Delta \tau_{23}$.
        This approximation is possible due to the approximation that the radius $r$ is constant during each time interval.\\

        At this point we can define a new time interval $\Delta \tau_{13}$ from point 1 to 3, which is equal to the sum of $\Delta \tau_{12}$ and $\Delta\tau_{23}$.
        The Schwarzschild line element for this new interval, is therefore equal to the sum of the line elements of the other intervals.\\

        By finding the derivative of the line element for $\Delta \tau_{13}$ with respect to $\phi$ and setting it equal to 0, we can find the change in coordinates, which maximizes the change of time coordinates.

    \subsection{Conclusion}\label{subsec:conclusion2}
        When doing the calculations, the terms not including $\phi$ will be constant.
        We now find that the quantity $r^2 \frac{d\phi}{d\tau}$ is the same in both time intervals, even with $d\phi$ and $d\tau$ changing.
        We can therefore conclude that this quantity must be conserved.

        This is an important result, as it tells us that the principle of maximum aging describes a path where the spin is constant.
        When in free fall, an object will therefore move on such a trajectory.\\

        This quantity can be compared to the spin in classical mechanics.
        In classical mechanics, spin is conserved if no outer force is applied, which corresponds to the trajectory of the object in free fall.




\section{Exercise 6}\label{sec:exercise-6}
    \subsection{Introduction}\label{subsec:introduction6}
    \subsection{Situation}\label{subsec:situation6}
    \subsection{Method}\label{subsec:method6}
    \subsection{Conclusion}\label{subsec:conclusion6}


\newpage
\section{Exercise 7}\label{sec:exercise-7}
    \subsection{Introduction}\label{subsec:introduction7}
    \subsection{Situation}\label{subsec:situation7}
    \subsection{Method}\label{subsec:method7}
    \subsection{Conclusion}\label{subsec:conclusion7}



\section{Appendix}\label{sec:appendix}
    \subsection{Exercise 1}\label{subsec:exercise-1}
        \subsubsection{Derivation of time interval difference}\label{subsec:derivation-of-time-interval-difference}
            We are here using a polar coordinate, to simplify calculations.
            Since the shell observer is at rest, and therefore not moving radially or tangentially, equation~\eqref{schwarzschild_line} can be simplified to
            \begin{align}
                \Delta s^2_{AB} &= \left(1-\frac{2M}{r}\right) \Delta t^2 - 0 - 0\\
                \Delta s^2_{AB} &= \left(1-\frac{2M}{r}\right) \Delta t^2
            \end{align}
            This can be rearranged to solve for the time interval $\Delta t$
            \begin{align}
                \Delta t = \frac{\Delta s_{AB}}{\sqrt{\left(1-\frac{2M}{r}\right)}} \label{eq1}
            \end{align}
            To be able to relate $\Delta t$ to $\Delta t_{shell}$, we will now need to express $\Delta s_{AB}$ in terms of $\Delta t_{shell}$.
            We know per definition that the time interval $\Delta t_{shell}$ is the time interval measured on the shell clock and therefore equates to the time interval for the proper time $\Delta \tau_{AB}$ on the shell.
            Furthermore, since the shell observer is at rest, we can relate $\Delta s_{AB}$ to $\Delta \tau_{AB}$.
            We find that
            \begin{align}
                \Delta s^2_{AB} &= (\Delta \tau_{AB})^2 - (\Delta x_{shell}_{AB})^2\\
                \Delta s_{AB} &= \Delta \tau_{AB}
            \end{align}
            This can then be used to define the relation $\Delta t_{shell} = \Delta s_{AB}$.
            Inserting this relation into~\eqref{eq1}, we find a formula which relates $\Delta t$ to $\Delta t_{shell}$.
            \begin{align}
                \Delta t = \frac{\Delta t_{shell}}{\sqrt{\left(1-\frac{2M}{r}\right)}}
            \end{align}

        \subsubsection{Exercise 1.3}\label{subsec:exercise-1.3}
            We us e a taylor series of degree 1 around $x = 0$.            
            \begin{align}
                T_1(x) = \left(\frac{1}{\sqrt{1-a}}-1 \right) + \frac{1}{2(1-a)^{3/2}}(x-a)\\
                T_1(x) = \left(\frac{1}{\sqrt{1-0}}-1 \right) + \frac{1}{2(1-0)^{3/2}}(x-0)\\
                T_1(x) = \left(\frac{1}{\sqrt{1}}-1 \right) + \frac{1}{2(1)^{3/2}}x\\
                T_1(x) = 0 + \frac{1}{2}x\\
                T_1\left(\frac{2M}{r}\right) = \frac{M}{r}
            \end{align}
\clearpage
\newpage
\printbibliography
\end{document}