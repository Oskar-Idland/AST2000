\documentclass[reprint,english,notitlepage]{revtex4-2}
\usepackage{amsmath}
\usepackage[mathletters]{ucs}
\usepackage[utf8x]{inputenc}
\usepackage[english]{babel}
\usepackage{esint}
\usepackage{physics,amssymb}
\usepackage{graphicx}
\usepackage{xcolor}
\usepackage{hyperref}
\usepackage{listings}
\usepackage{subfigure}
\hypersetup{
    colorlinks,
    linkcolor={red!50!black},
    citecolor={blue!50!black},
    urlcolor={blue!80!black}}

\lstset{inputpath=,
	backgroundcolor=\color{white!88!black},
	basicstyle={\ttfamily\scriptsize},
	commentstyle=\color{magenta},
	language=Python,
	morekeywords={True,False},
	tabsize=4,
	stringstyle=\color{green!55!black},
	frame=single,
	keywordstyle=\color{blue},
	showstringspaces=false,
	columns=fullflexible,
	keepspaces=true}

\begin{document}
\title{Simulation of a gas-driven rocket engine}
\author{Oskar Idland \& Jannik Eschler}
\date{\today}
\affiliation{Institute of Theoretical Astrophysics, University of Oslo}

\begin{abstract}
This is an abstract \colorbox{red}{Complete this summary at the end of the paper}
\end{abstract}
\maketitle

\section{Introduction}
Humanity is the only known species to use and shape everything from elements to their entire planet to their benefit.
Now, the next logical step is to look further and find out what there can be found beyond our home planet.
There are different ways to get to, and explore space, but the most established way to do this is by using rockets.\\
Rocket engines, while highly advanced, still simply utilize Newton's laws of motion to create thrust.
More specifically the third law, which states "For every action, there is an equal and opposite reaction".
When wanting to propel our rocket engine as fast as possible upwards into space and being able to maneuver, we are therefore required
to be able to expel the right amount of matter at the right time to create the force needed to propel the rocket as wanted.\\
The first step to creating such an engine is simulating both its inner workings and the engine as part of a rocket during a launch to determine engine-, rocket- and launch-parameters
as well as gaining a better understanding of the complex system, which makes up a functional rocket.
The rocket in this paper will be using hot $ H_2 $ gas under high pressure which will be expelled out the end of the rocket engine to create a force.
This is due to $H_2$ being an ideal gas, which simplifies calculations. \colorbox{red}{???(Write more on the simulation and theory)???}

\section{Theory}
To complete the calculations for the engine we are going to use a lot of statistics to simplify the behavior of the gas particles. 

\section{Method}
As we are trying to simulate a rocket engine, using gas being expelled at high speed, it is important to be able to simulate and understand the gas.
Gases are comprised of many individual elementary particles, or in the case of Hydrogen gas molecules.


\section{Results}
\section{Discussion}
\section{Conclusion}
\section{References}
\section{Appendix: Mathematical Derivations}
	\subsection*{Challenge A.3.1}
	\[
	\left< v \right> = ∫  _{0}^{\infty} vP(v) \ \mathrm{d}v
	\]
	\[
	\int_{0}^{\infty} x^{\frac{3}{2}} e^{-x} \mathrm{d}x = \frac{3}{4} \sqrt{\pi}  
	\]
	We try to rewrite the first integral to fit the form of the second 
	\[
	\int _{0}^{\infty} vP(v) \ \mathrm{d}v = \int _{0}^{\infty} \left( \frac{m}{2 \pi k T} \right) ^{\frac{3}{2}} e ^{-\frac{1}{2} \frac{mv^{2}}{k T}} 4 \pi v^{3}
	\]\newline 
	\[
	\frac{4}{\sqrt{\pi}} \int _{0}^{\infty} \left( \frac{mv^{2}}{2 \pi k T} \right) ^{\frac{3}{2}} e ^{-\frac{1}{2} \frac{mv^{2}}{k T}} 
	\]
	We use substitution to get the aforementioned integral
	\[
	\frac{4}{\sqrt{\pi}} \frac{3}{4} \sqrt{\pi} = 3
	\]


\end{document}