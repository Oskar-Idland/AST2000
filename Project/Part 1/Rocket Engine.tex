\documentclass[reprint,english,notitlepage]{revtex4-2}
\usepackage{amsmath}
\usepackage[mathletters]{ucs}
\usepackage[utf8x]{inputenc}
\usepackage[english]{babel}
\usepackage[mathletters]{ucs}
\usepackage{physics,amssymb}
\usepackage{graphicx}
\usepackage{xcolor}
\usepackage{hyperref}
\usepackage{listings}
\usepackage{subfigure}
\hypersetup{
    colorlinks,
    linkcolor={red!50!black},
    citecolor={blue!50!black},
    urlcolor={blue!80!black}}

\lstset{inputpath=,
	backgroundcolor=\color{white!88!black},
	basicstyle={\ttfamily\scriptsize},
	commentstyle=\color{magenta},
	language=Python,
	morekeywords={True,False},
	tabsize=4,
	stringstyle=\color{green!55!black},
	frame=single,
	keywordstyle=\color{blue},
	showstringspaces=false,
	columns=fullflexible,
	keepspaces=true}

\begin{document}
\title{Simulation of a gas-driven rocket engine}
\author{Oskar Idland \& Jannik Eschler}
\date{\today}
\affiliation{Institute of Theoretical Astrophysics, University of Oslo,
\\P.O. Box 1029 Blindern, 0315 Oslo}

\begin{abstract}
This is an abstract \colorbox{red}{Complete this summary at the end of the paper}
\end{abstract}
\maketitle

\section{Introduction}
	Rocket engines, while highly advanced, still simply utilize Newton's laws of motion. More specifically the third law, which states "For every action, there is an equal and opposite reaction". We want to propel our rocket engine as fast as possible upwards into space.
	Our rocket will be filled with hot $ H_2 $ gas under high pressure which we will expel out the end of the rocket engine. \colorbox{red}{Write more on the simulation and theory}
\section{Theory}
To complete the calculations for the engine we are going to use a lot of statistics to simplify the behavior of the gas particles. 
\section{Method}

	\subsection*{Challenge: 3.1}
		\[
		\left< v \right> = ∫  _{0}^{\infty} vP(v) \ \mathrm{d}v
		\]
		\[
		\int_{0}^{\infty} x^{\frac{3}{2}} e^{-x} \mathrm{d}x = \frac{3}{4} \sqrt{\pi}  
		\]
		We try to rewrite the first integral to fit the form of the second 
		\[
		\int _{0}^{\infty} vP(v) \ \mathrm{d}v = \int _{0}^{\infty} \left( \frac{m}{2 \pi k T} \right) ^{\frac{3}{2}} e ^{-\frac{1}{2} \frac{mv^{2}}{k T}} 4 \pi v^{3}
		\]
		\[
		\frac{4}{\sqrt{\pi}} \int _{0}^{\infty} \left( \frac{mv^{2}}{2 \pi k T} \right) ^{\frac{3}{2}} e ^{-\frac{1}{2} \frac{mv^{2}}{k T}} 
		\]
		We use substitution to get the aforementioned integral
		\[
		\frac{4}{\sqrt{\pi}} \frac{3}{4} \sqrt{\pi} = 3
		\]
	  

\section{Results}
\section{Discussion}
\section{Conclusion}
\section{References}



\end{document}