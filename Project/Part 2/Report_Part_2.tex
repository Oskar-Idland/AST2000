%! Author = jannikeschler
%! Date = 18/09/2022

\documentclass[reprint,english,notitlepage]{revtex4-2}
\usepackage{amsmath}
\usepackage[mathletters]{ucs}
\usepackage[utf8x]{inputenc}
\usepackage[english]{babel}
\usepackage{esint}
\usepackage{physics,amssymb}
\usepackage{graphicx}
\usepackage{xcolor}
\usepackage{hyperref}
\usepackage{listings}
\usepackage{subfigure}
\hypersetup{
    colorlinks,
    linkcolor={red!50!black},
    citecolor={blue!50!black},
    urlcolor={blue!80!black}}

\lstset{inputpath=,
	backgroundcolor=\color{white!88!black},
	basicstyle={\ttfamily\scriptsize},
	commentstyle=\color{magenta},
	language=Python,
	morekeywords={True,False},
	tabsize=4,
	stringstyle=\color{green!55!black},
	frame=single,
	keywordstyle=\color{blue},
	showstringspaces=false,
	columns=fullflexible,
	keepspaces=true}

\begin{document}
\title{Simulating Planetary orbits of a solar system}
\author{Jannik Eschler & Oskar Idland}
\date{\today}
\affiliation{Institute of Theoretical Astrophysics, University of Oslo}

\begin{abstract}
This is an abstract \colorbox{red}{Complete this summary at the end of the paper}
\end{abstract}
\maketitle

\section{Introduction}
When making our way out of the earth's atmosphere, we will have to know where our destination will be after the launch.
To be able to reach our destination we therefore have to get known in our solar system and explore and simulate the different planets in the system.
After finding out the orbits of the planets, it will be much easier to coordinate our launch to our target to minimize the time and effort required to find the destination.
Using this, we will also be able to give estimates about the possibility of extraterrestrial life discovering our solar system.

\section{Method}
Method of our part goes here

\section{Results}
This section is for results only
	\subsection{Subsection}
    Subsection?

\section{Discussion}
Here comes the discussion :)
	\subsection{Subsection}

\section{Conclusion}
And the conclusion


\section{References}
References are important!

\section{Appendix: Mathematical Derivations}
This is the appendix

\end{document}